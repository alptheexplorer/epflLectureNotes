  
\documentclass[titlepage]{article}
\usepackage[utf8]{inputenc}
\usepackage{amsmath}
\usepackage{tcolorbox}
\usepackage{amssymb}
\usepackage{amsthm}
\usepackage{empheq}
\usepackage{xcolor}
\usepackage{float}
\usepackage[
top    = 2.50cm,
bottom = 2.50cm,
left   = 2.75cm,
right  = 2.75cm]{geometry}
\usepackage{fancyhdr}
\pagestyle{fancy}
\lhead{Analysis 2}
\rhead{EPFL/Alp Ozen}

\newtheorem{remark}{Remark}[section]
\newtheorem{theorem}{Theorem}[section]
\newtheorem{prop}{[Proposition]}
\newtheorem{definition}{Definition}
\newtheorem{question}{Question}

\newcommand{\interior}[1]{%
  {\kern0pt#1}^{\mathrm{o}}%
}
\newcommand{\Rn}{\mathbb{R}^n}
\newcommand{\Rm}{\mathbb{R}^m}
\newcommand{\R}{\mathbb{R}}

\title{\textbf{Analysis 2 - Thomas Mountford}}
\author{Alp Ozen}
\date{Spring 2019}
\newtheorem{example}{Example}[section]
\newtheorem{axiom}{Axiom}
\newtheorem{cor}{Corollary}

\begin{document}

\maketitle
\tableofcontents
\clearpage


\section{Notions on \Rn }

Let's recall that $\Rn$ is a Euclidean vector space. We define a scalar product on $\Rn$ as follows:

\begin{definition}
\begin{equation*}
    < \cdot, \cdot > : \Rn \times \Rn \to \mathbb{R}
\end{equation*}


\begin{enumerate}
    \item $<x,x> \geq 0$
    \item $<x,y> = <y,x>$
    \item $<ax + by, z> = a<x,z> + b<y,z>$
\end{enumerate}

\end{definition}

\subsection{Introducing topological properties on \Rn}
A \textit{norm} is defined as function that maps some real vector space $E$ to $\mathbb{R}$ and satisfies:

\begin{align*}
    1) \ |x| \geq 0 \ \forall x \in E, \ |x| = 0 \iff x = 0\\
    2) \ |\lambda \cdot x| = |\lambda| \cdot |x| \\
    3) \ |x+y| \leq |x| + |y|
\end{align*}

In our intuitive understanding of $\Rn$ we are actually thinking about the Euclidian space $\Rn$ equipped with the Euclidian norm.

\begin{definition}\textbf{Euclidian norm}
$$|x|_{2} = \sqrt{<x,x>} = (\sum^{n}_{i}x^{2}_{k} )^{\frac{1}{2}} $$
\end{definition}

And from this naturally follows the definition of Euclidian distance:
\begin{definition}
$$d(x,y) = |x-y|$$
\end{definition}

We note that $d$ satisfies the same 3 properties as the norm. Thus, the couple \textbf{$(E,d)$} is called a metric space. 
\\

And now more definitions:

\begin{definition} \textbf{Open sets}
\begin{enumerate}
    \\
    
    \item \textbf{Open ball} $ B(a,r) := \{x \in \Rn : d(x,a) < r\}$
    \item \textbf{Open subset} Some subset $S \subset \Rn$ is open if $\forall x \in \Rn, \ \exists \epsilon > 0 \ B(x,\epsilon) \subset S$
    \item \textbf{Closed subset} Some $S$ is closed if $\Rn - S$ is open, note that the empty set and $\Rn$ are both open and closed. 
    \item \textbf{The interior and boundary of a set} $a$ is in the interior of $S$ if $\exists \epsilon > 0 \ B(a,\epsilon) \subset S$ and $b$ is in the boundary of a set $S$ if any $B(a, \epsilon)$ contains points from both $S$ and $\Rn - S$. The set of all interior points is denoted $\interior(S)$ and set of all boundary points is denoted $\partial S$
    \item \textbf{Closure of a set} $a$ is a closure of $S$ if for any $B(a,\epsilon)$ we have $B(a, \epsilon) \cap S \not = \emptyset$ 
\end{enumerate}

\end{definition}

\begin{definition}\textbf{Topology}

A Topology exists whenever the following are satisfied:

$$\text{For a given} \ M \subset \Rn \text{we define} \ O \subset P(M) $$
\begin{enumerate}
    \item $\emptyset \in O \ , M \in O $
    \item $U \in O, V \in O \rightarrow U \cap V \in O$
    \item $U_{\alpha} \in O \rightarrow \bigcup_{\alpha} U_{\alpha} \in O$
\end{enumerate}

\end{definition}


\begin{definition}\textbf{Closure of a set}
A point $a \in \Rn$ is a closure point of $S$ if for any $B(a,\epsilon)$ we have:
$$B(a,\epsilon) \cap S \not = \emptyset$$
The set of all closure points called the closure of $S$ is denoted $\bar{S}$
\end{definition}

\begin{theorem}
Important results on closures and boundaries
\begin{align*}
    \interior{S} \subset S \subset \bar{S}\\
    \bar{S} = \interior{S} \cup \partial S\\
    S \ \text{is open iff} \ S = \interior{S}\\
    S \ \text{is closed iff} \ S = \bar{S}\\
\end{align*}
    
\end{theorem}

\begin{definition} \textbf{Sequence in \Rn}
\\

$f : \mathbb{N} \to \Rn$ such that:
$$x_{k} = f(k) \in \Rn \forall k \in \mathbb{N}$$

\end{definition}


\begin{definition} \textbf{Convergence}
\\

The definition of convergence in $\Rn$ is very similar to its counterpart in $\mathbb{R}$. We say that $x_{k}$ converges to $x$ iff:

$$ \forall \epsilon > 0 \ \exists N_{\epsilon} \ \forall k \geq N_{\epsilon} \ d_{2}(x_{k},x}) < \epsilon$$
\end{definition}


\begin{definition} \textbf{Complete spaces aka. Banach spaces}
\\
A space is called complete if every Cauchy sequence in this space converges to a limit. Some example of complete spaces are $\mathbb{R}, \Rn$
\end{definition}

We restate the axioms of a \textbf{Norm} and of a \textbf{Metric}. The subtle difference between the two is that a norm can only be applied to some vector space whereas a metric is applicable to other spaces.
\begin{remark} \textbf{Axioms of norm and metric}
\\
\textbf{Norm:}
\begin{enumerate}
    \item $||x|| \geq 0$
    \item $ ||\lambda x|| = |\lambda| ||x||$
    \item $||x+y|| \leq ||x|| + ||y||$
\end{enumerate}
\end{remark}
\\

\textbf{Metric:}
\begin{enumerate}
    \item $d(x,y) \geq 0$
    \item $d(x,y) = d(y,x)$
    \item $d(x,y) \leq d(x,z) + d(z,y)$
\end{enumerate}

\end{document} 
